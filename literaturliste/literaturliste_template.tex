\documentclass[12pt,a4paper]{report}
\usepackage[utf8]{inputenc}
\usepackage[ngerman]{babel}
\usepackage[T1]{fontenc}
\usepackage{lmodern}
\usepackage[left=2cm,right=2cm,top=2cm,bottom=2cm]{geometry}
\usepackage{bibgerm}
\usepackage{hyperref}

%\usepackage{etoolbox}
%\patchcmd{\thebibliography}{\chapter*}{\section*}{}{}

% No indent @ line start
\parindent 0pt

% The bibstyle
% gerplain is for only numbers in alphabetic order
% geralpha is for name+year in alphabetic order
\bibliographystyle{gerplain}

% Change the text from the list of listings
\addto\captionsngerman{\renewcommand{\bibname}{}}

% A text with quotation marks
% @par1: The text you want to quote
% »text«
\newcommand*{\QuoteM}[1]{\frqq #1\flqq}

\begin{document}

%TODO Filename: Vortrag_Agile_Vorgehensmodelle.pdf
\section*{Literaturliste}
Dies ist die Literaturliste für den Vortrag zum Thema \QuoteM{Agile 
Vorgehensmodelle} am 14.04.2016 im Fach \QuoteM{Spezielle Gebiete zum Software 
Engineering}. Die Ebooks sind via VPN oder im Netzwerk der FH Bielefeld verfügbar.

\subsection*{Testgetriebene Entwicklung}
\subsubsection*{Pflicht}
Kapitel 5: S. 67, S. 70--72: \cite{TestingPython}
\subsubsection*{Extra}
Kapitel 5: S. 74--83: \cite{TestingPython}

\subsection*{Kanban}

\subsubsection*{Pflicht}
Für eher audiovisuelle Lerner (über VPN auch ohne Account): \cite{KanbanV2B}\\
Besonders wichtig sind folgende Videos:

\begin{itemize}
	\item \QuoteM{Was ist Kanban?}
	\item \QuoteM{Überblick über den Kanban-Prozess}
	\item \QuoteM{Kanban in der IT}
	\item \QuoteM{Vorteile von Kanban}
	\item \QuoteM{Theorie hinter Kanban}
\end{itemize}

Kapitel 3: S. 17--22 \cite{KanbaninderIT}
\subsubsection*{Extra}
Kapitel 3: S. 14--17 \cite{KanbaninderIT}

\subsection*{Scrum}
\subsubsection*{Pflicht}
Der gültige Leitfaden für Scrum, von den Entwicklern. \\
Enthält die Basics, in unterschiedlichen Sprachen: \cite{ScrumGuide}
\subsubsection*{Extra}
Besonders Kapitel 2.2: ab S. 10: \cite{AgileITProjekte} \\
Besonders Kapitel 5: ab S. 61: \cite{AgileProzesse} \\
Besonders Kapitel 4: ab S. 81: \cite{ScrumSystem} \\
\nocite{ScrumUnternehmenspraxis}
Englischsprachig, setzt Grundwissen voraus: \cite{ScrumCulture}\\
Für eher audiovisuelle Lerner (über VPN auch ohne Account): \cite{VideoToBrain} 
\bibliography{Literature}
\end{document}